\documentclass{article}
\usepackage{graphicx} % Required for inserting images
\usepackage{amsmath}
\usepackage{amssymb}
\usepackage{amsthm}

\title{Math 251 - Exercise Set 2 Zehra's Solutions}
\begin{document}

\maketitle
% \tableofcontents
\section{Theorems and Definitions that are used in the solutions} 
\begin{itemize}
    \item  $p \in \partial S$. $\iff$ for every $r > 0$, $B(p,r) \cap S \neq \emptyset$ and $B(p,r) \cap (\mathbb{R}^m \setminus S) \neq \emptyset$. \\

\end{itemize}

\section{Solutions}

\begin{enumerate} \boldmath
    \item Let 
    \item Let A(1,4) and B(5,7) be two points in the plane. Find two open sets U and V s.t.
$A \in U$, $B \in V$ and $U \cap V = \emptyset$\\ 

\textbf{Solution:} \\

Let $U = B(A,1)$ and $V = B(B,1)$. \\
Then $A \in U$, $B \in V$ and $U \cap V = \emptyset$.\\


    \item  Let $S \subseteq \mathbb{R}^m$.\\
    \begin{enumerate}

        \item  Show that
$p \in \partial S$ iff there exists sequences $\{p_n\}$ in $S$ with $\lim_{n\to \infty}  p_n = p$ and $\{q_n\}$ in $\mathbb{R}^m \setminus S$ with $\lim_{n\to \infty} q_n = p$. \\

\textbf{Proof:} \\

$\Rightarrow$ Suppose $p \in \partial S$. \\
Then for every $r > 0$, $B(p,r) \cap S \neq \emptyset$ and $B(p,r) \cap (\mathbb{R}^m \setminus S) \neq \emptyset$. \\
Let $n \in \mathbb{N}$. \\
Then there exists $p_n \in B(p, \frac{1}{n}) \cap S$ and $q_n \in B(p, \frac{1}{n}) \cap (\mathbb{R}^m \setminus S)$. \\ 
Then $\lim_{n\to \infty} p_n = p$ and $\lim_{n\to \infty} q_n = p$.\\

$\Leftarrow$ Suppose there exists sequences $\{p_n\}$ in $S$ with $\lim_{n\to \infty}  p_n = p$ and $\{q_n\}$ in $\mathbb{R}^m \setminus S$ with $\lim_{n\to \infty} q_n = p$. \\ Then for every $r > 0$, there exists $N \in \mathbb{N}$ such that $p_n \in B(p,r)$ for all $n \geq N$ and $q_n \in B(p,r)$ for all $n \geq N$. \\ Then $B(p,r) \cap S \neq \emptyset$ and $B(p,r) \cap (\mathbb{R}^m \setminus S) \neq \emptyset$. \\ Thus $p \in \partial S$.\\

    \item Show that $S$ is closed iff the limit of every convergent sequence $\{p_n\}$ in $S$ belongs to $S$.\\

\textbf{Proof:} \\

$\Rightarrow$ Suppose S is closed. \\
Let $\{p_n\}$ be a convergent sequence in S with $\lim_{n\to \infty} p_n = p$. \\
Then for every $r > 0$, there exists $N \in \mathbb{N}$ such that $p_n \in B(p,r)$ for all $n \geq N$. \\
Then $B(p,r) \cap S \neq \emptyset$. \\
Thus $p \in \overline{S} = S$.\\

$\Leftarrow$ Suppose the limit of every convergent sequence $\{p_n\}$ in S belongs to S. \\
Let $p \in \overline{S}$. \\
Then there exists a sequence $\{p_n\}$ in S with $\lim_{n\to \infty} p_n = p$. \\
Then $p \in S$ because the limit of every convergent sequence $\{p_n\}$ in S belongs to S. \\
Thus $\overline{S} = S$.\\
    \end{enumerate} 
\item Find $\limsup x_n$ and $\liminf x_n$ if $\{x_n\}$ is
\begin{enumerate}
    \item $\{2,4,2,4,6,2,4,8,2,4,10,\dots \}$,\\

    \textbf{Solution:} \\ 
    $\limsup x_n = 6$ and $\liminf x_n = 2$.\\
    \item $\{1,0,-1,1,0,-1,1,0,-1,\dots \},$\\

    \textbf{Solution:} \\ 
    $\limsup x_n = 1$ and $\liminf x_n = -1$.\\
    \item $\{-2,1,1,-2,1,1/2,-2,1,1/3,-2,1,1/4,-2,1,1/5, \dots\},$\\
    
    \textbf{Solution:} \\ 
    $\limsup x_n = 1$ and $\liminf x_n = -2$.\\
    \item $\{\frac{3n - 1}{2n}\}_{n=1}$,\\
    
    \textbf{Solution:} \\
    $\limsup x_n = \frac{3}{2}$ and $\liminf x_n = \frac{1}{2}$.\\
    \item $\{0,1,0,3,0,5,0,7,0,\dots \}$,\\
    
    \textbf{Solution:} \\
    $\limsup x_n = 7$ and $\liminf x_n = 0$.\\
    \item $x_n = -3n+2$ for $n \geq 1$,\\
    
    \textbf{Solution:} \\
    $\limsup x_n = 2$ and $\liminf x_n = -\infty$.\\

\end{enumerate}
\item Find a sequence $\{a_n\}$ with $\limsup a_n = 3$ and $\liminf a_n = -2$.\\

\textbf{Solution:} \\
Let $a_n = (-1)^n n$ for $n \geq 1$. \\
Then $\limsup a_n = 3$ and $\liminf a_n = -2$.\\
\item  Let $S = \{x \in \mathbb{R} | x^3 + 3x^2 + x + 3 < 0\}$. Find $\sup(S)$. Is $S$ bounded below?\\

\textbf{Solution:} \\
\textbf{sup(S)} = 0. Since $x^3 + 3x^2 + x + 3 = (x+1)(x^2+2x+3) = (x+1)(x+1+i\sqrt{2})(x+1-i\sqrt{2})$, $x^3 + 3x^2 + x + 3 < 0$ if and only if $x \in (-1,0)$.\\
$S$ is bounded below. Because $-1$ is a lower bound of $S$.\\
????   


\item Is the sequence ${x_n} = \{\frac{\sin(n)}{n^3}\}$ a Cauchy sequence?\\

\textbf{Solution:} \\
Yes, the sequence ${x_n} = \{\frac{\sin(n)}{n^3}\}$ is a Cauchy sequence.\\

\item Let $A \subseteq \mathbb{R}^n$ and $p \in \mathbb{R}^n$. Show that $p \in \bar A \Leftrightarrow B(p,\varepsilon) \cap A \neq \emptyset$ for all $\varepsilon > 0$.\\

\textbf{Proof:} \\
$\Rightarrow$ Suppose $p \in \bar A$. \\
Then for every $\varepsilon > 0$, $B(p,\varepsilon) \cap A \neq \emptyset$.\\




\end{enumerate}


\end{document}